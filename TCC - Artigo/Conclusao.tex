%\thispagestyle{empty}
\section{CONCLUSÃO}
Este trabalho apresentou, inicialmente, uma introdução sobre a importância da formalização de trabalhos acadêmicos de acordo com as normas específicadas em seus regulamentos e a dificuldade dos alunos de seguirem estas normas. Como solução para este problema, propõe-se a utilização do LaTeX para elaborar os modelos de documentos de trabalhos acadêmicos, visando facilitar a formatação dos mesmos.

A elaboração deste trabalho foi realizada seguindo as seguintes etapas:

•	Estudo dos mecanismos de produção de documentos em LaTeX;

•	Desenvolvimento do modelo de relatório de estágio do IFTM no LaTeX;

•	Desenvolvimento do modelo de projeto de pesquisa (projeto de TCC) do IFTM no LaTeX;

•	Desenvolvimento do modelo de artigo do IFTM no LaTeX;

•	Desenvolvimento do modelo da monografia do IFTM no LaTeX;\\

Todos os modelos abordados neste trabalho foram implementados em LaTeX e encontram-se dísponíveis em um repositório público do GitHub, prontos para serem utilizados pelos estudantes do IFTM. O modelo da monografia foi utilizado para gerar este documento. 

Dessa forma, pode-se concluir que, os objetivos traçados inicialmente foram cumpridos e que, através do uso dos modelos elaborados é possível facilitar a formatação de trabalhos acadêmicos.
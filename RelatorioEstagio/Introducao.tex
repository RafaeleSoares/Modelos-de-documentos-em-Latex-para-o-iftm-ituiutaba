\newpage
\section{INTRODUÇÃO}

\pagenumbering{arabic}

\indent Na atualidade, softwares são utilizados em praticamente todas as áreas da sociedade, seja em sistemas simples ou em aplicações sofisticadas e complexas. Assegurar a qualidade do software é imprescindível para garantir a qualidade do produto. O teste do software é um elemento vital, pois permite a identificação de erros durante o desenvolvimento. Estes testes, geralmente, são realizados manualmente por equipes de teste, ou mesmo pelos desenvolvedores e é comum que erros sejam encontrados durante o processo de desenvolvimento, tornando-se necessário corrigi-los e refazer os testes manuais. Porém, a execução manual repetitiva de um conjunto de testes pode se tornar uma atividade trabalhosa e exaustiva.

Uma alternativa envolve a automação de testes. O uso dessa técnica vem crescendo cada vez mais, pois possibilita a realização de testes repetidas vezes, de forma rápida e prática e com menor probabilidade de erro. \nocite{Bernardo2011}

No decorrer deste relatório será relatada a experiência de aprendizado vivida pela estagiária através da descrição das atividades realizadas na área de automação de testes.

\subsection{OBJETIVO GERAL}
Construir testes automatizados usando os Frameworks jUnit e Selenium para sistema de vendas da empresa VIVO.

\subsection{OBJETIVOS ESPECÍFICOS}
Os pontos listados abaixo foram necessários para alcançar o objetivo geral, sendo que cada um deles possibilitaram ao estagiário experiências variadas e que ainda não haviam sido vivenciadas durante o curso até então:

•	Aprendizado e utilização da ferramenta Selenium WebDriver;

•	Aprendizado e utilização da ferramenta e-fa3;

•	Utilização do framework JUnit;

•	Utilização de um editor de planilhas (Excel);

•	Utilização da linguagem Java;

•	Utilização da IDE eclipse;

•	Criação de scripts de teste referentes aos casos de teste solicitados;

\subsection{JUSTIFICATIVA}
A cada modificação realizada no software, é necessário testar todo o sistema novamente, e não apenas o módulo que foi modificado. Automatizar testes evita um trabalho manual repetitivo, economizando tempo e recursos e assegurando uma maior qualidade ao software.

A automação é feita através de scripts simples, que com a ajuda de frameworks, como o JUnit, verificam o sistema automaticamente sem a necessidade da intervenção humana durante a verificação, podendo executar os testes quantas vezes forem necessárias.
%\thispagestyle{empty}
\section{ATIVIDADES EXECUTADAS}

%\pagestyle{headings}
%\pagenumbering{arabic}

Primeiramente foi ministrado um treinamento de duas semanas por profissionais do setor de automação de testes da empresa Everis Brasil, que nos passaram conhecimentos à cerca das ferramentas utilizadas por este setor, o Selenium WebDriver e o $e-FA^{3}$.

Na primeira semana do treinamento, os profissionais da Everis explicaram os comandos do Selenium WebDriver e passaram exercícios para automação de testes em webpages criadas para o treinamento. Na segunda semana, os estagiários continuaram a fazer atividades utilizando o Selenium WebDriver e no último dia de treinamento foi passado como funciona os comandos do $e-FA^{3}$.

Como a primeira semana do estágio iniciou-se em uma quarta feira, houve apenas três dias úteis. No primeiro dia foi passado a todos os estagiários o mesmo caso de teste, sendo que este já havia sido feito antes pelos funcionários da área de automação de testes da Everis. O teste consistia em validar a criação de um Lead do tipo Agendamento no sistema do cliente. Como os estagiários não possuem conhecimento acerca do que faz cada módulo do sistema, o cliente detalhou passo a passo o que deve ser feito para que essa validação fosse feita com sucesso, portanto os estagiários executavam o caso de teste manualmente, ao menos uma vez, e mapeavam os web elements necessários para a criação do script.

Nos dois dias seguintes da primeira semana, foram passados casos de testes diferentes. Por não ter casos de testes suficientes para que cada estagiário desenvolvesse um script diferente, os casos de testes foram divididos para grupos de estagiários, porém cada estagiário desenvolvia individualmente um script para o caso de teste correspondente ao seu grupo. Para cada grupo foi nomeado um estagiário como monitor para auxiliar caso alguém no grupo apresentasse dificuldades. A estagiária ficou no grupo do case de teste dezenove (CT19), cujo estagiário Danton era o monitor. O CT19 consistia em validar uma cotação de pedido no sistema da Vivo.

Na segunda semana do estágio, devido a instabilidade do servidor, foi decidido que cada grupo passaria a desenvolver apenas um script de teste e não mais individualmente, assim diminuiria o número de solicitações de acesso ao servidor, portanto coube ao monitor dividir as tarefas do caso de teste designado ao grupo. A principal função da estagiária no CT19 foi auxiliar na criação do script de teste, fazer o mapeamento dos web elements necessários para a realização do caso de teste e testar manualmente o caso de teste dezenove, anotando passo a passo as etapas para que o teste fosse executado com sucesso.

A terceira semana do estágio foi a continuação da elaboração do script de teste para o CT19, porém foi necessário refazer alguns mapeamentos, devido a mudanças no sistema, e repetir o teste manual, devido a alteração da massa de dados passada pelo cliente.

Durante a quarta semana foram passados casos de testes em um sistema diferente, porém da mesma empresa (Vivo), para os estagiários que estavam no laboratório B04. Todos os casos de testes já haviam sidos feitos anteriormente pelos profissionais da Everis. A estagiária ficou responsável pelo caso de teste excluir área tarifaria e pelo caso de teste incluir bairro. Este sistema foi mais simples de se trabalhar, pois não apresentava instabilidade.

A quinta semana do estágio foi a junção dos casos de teste do mesmo módulo do sistema. O módulo área tarifária possuía quatro casos de teste, o de incluir área tarifária, o de consultar área tarifária, o de alterar área tarifária e o de excluir área tarifária. Como a estagiária ficou responsável pelo caso de teste excluir área tarifária, foi necessário reunir-se com os estagiários responsáveis pelos outros três casos de teste deste módulo para juntá-los em um único script. O módulo bairro também possuía quatro casos de teste, o de incluir bairro, o de consultar bairro, o de alterar bairro e o de excluir bairro. Neste módulo também foi necessário unir os quatro casos de teste do módulo, portanto a estagiária reuniu-se com os estagiários responsáveis pelos outros três casos de teste deste módulo.
\newpage
%\thispagestyle{empty}
\section{INTRODUÇÃO}

\pagenumbering{arabic}

\indent Atualmente, além da matriz curricular especificada, a maioria dos cursos de graduação exigem outros requisitos para sua conclusão, tais como: o estágio supervisionado e o Trabalho de Conclusão de Curso (TCC).

O estágio é um importante instrumento para a formação profissional. É através do estágio que o estudante tem a oportunidade de entrar em contato direto com o ambiente profissional no qual será inserido. Além possibilitar a aplicação prática do conhecimento teórico adquirido ao longo da graduação \cite{Piccinini2012}.

Já o TCC permite que o universitário mostre o que aprendeu no decorrer de sua vida acadêmica. Além disso, permite a contribuição do aluno para o avanço científico e tecnológico \cite{Pereira2009}, pois é neste trabalho que o aluno será capaz de apresentar soluções para um determinado problema da área estudada, como também o desenvolvimento de novas abordagens.

Tanto o TCC como o estágio devem ser elaborados de acordo com seus respectivos regulamentos, que variam de acordo com a instituição. O Instituto Federal de Educação, Ciência e Tecnologia do Triangulo Mineiro (IFTM) possui manuais para a normatização destes documentos acadêmicos, que são conduzidos por normas definidas pela Associação Brasileira de Normas Técnicas (ABNT) \cite{manualTCC,manualEstagio}. Porém, alguns alunos encontram dificuldades em utilizar os manuais disponibilizados pelo IFTM durante a elaboração de seus respectivos trabalhos.

O LaTeX é um sistema de preparação de documentos para composição tipográfica de alta qualidade \cite{LATEX2018}, ou seja, o LaTeX possui um conjunto de comandos que facilita a produção de documentos acadêmicos, pois incentiva os autores a não se preocuparem muito com a aparência de seus documentos, mas a se concentrarem em seu conteúdo. Portanto, para tentar sanar as dificuldades dos estudantes em relação à elaboração de seus trabalhos acadêmicos, o LaTeX pode ser utilizado como uma ferramenta para facilitar a elaboração dos trabalhos acadêmicos.

\subsection{OBJETIVO GERAL}
Desenvolver e disponibilizar modelos dos documentos acadêmicos do IFTM em LaTeX. O conteúdo deste documento juntamente ao seu código fonte tem o objetivo de ser um modelo de trabalho acadêmico e um tutorial para utilização do LaTeX.

\subsection{OBJETIVOS ESPECÍFICOS}
•	Estudar os mecanismos de produção de documentos no LaTeX;

•	Desenvolver o modelo de relatório de estágio;

•	Desenvolver o modelo de projeto de pesquisa (projeto de TCC);

•	Desenvolver o modelo de TCC – artigo;

•	Desenvolver o modelo de TCC – monografia;

•	Elaborar um manual de uso desses modelos.
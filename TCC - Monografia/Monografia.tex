\documentclass[12pt]{article}

\usepackage[utf8]{inputenc}
\usepackage[brazil]{babel}
\usepackage{graphicx} %pacote para inserir figuras
\usepackage{geometry} %pacote que pertmite dimensionar a página
\usepackage{setspace} %pacote que pertmite alterar epaçamento entre linhas
\usepackage{indentfirst} %indentação de parágrafos
\usepackage{times} %fonte Times New Roman
\usepackage{sectsty} %pacote que pertmite alterar o estilo de seções e sub-seções
\usepackage{listings} %pacote para inserir código fonte
\usepackage[alf]{abntex2cite} %pacote de estilos bibliográficos da abnt

%alterando as margens para os padrões da abnt
%esquerda 3cm, superior 3cm, direita 2cm, inferior 2cm
\geometry{a4paper,left=3cm,top=3cm,right=2cm,bottom=2cm}

%definindo o espaçamento 1.5 como padrão do documento
\setstretch{1.5}

%alterando o estilo das seções e sub-seções
\sectionfont{\fontsize{12}{15}\selectfont}
\subsectionfont{\fontsize{12}{15}\normalfont}
\subsubsectionfont{\fontsize{12}{15}\selectfont}

%alterando a legenda das listagens
\addto\captionsbrazil{
\renewcommand{\lstlistingname}{Listagem}
}

%configurações para inserir códigos fonte
\lstset{
  numbers=left, %números das linhas a esquerda
  numberstyle=\tiny, %alerando o tamanho dos números das linhas
  basicstyle=\footnotesize, %alterando o tamanho da fonte
  stringstyle=\ttfamily, %alterando o estilo das strings que existem no código fonte
  showstringspaces=false, % não sublinhar espaços em branco
  captionpos=b, %posição da legenda
  frame=tb %adicionar bordas
}
\renewcommand{\lstlistlistingname}{Lista de Listagens}

\begin{document}
\input Capa
\input FolhaRosto
\input FolhaAprovacao
\input Dedicatoria
\input Agradecimento
\input Resumo
\input Abstract

%lista de figuras
\newpage
\listoffigures
\thispagestyle{empty}

%lista de tabelas
\newpage
\listoftables
\thispagestyle{empty}

%lista de listagens
\newpage
\lstlistoflistings
\thispagestyle{empty}

%sumário
\newpage
\thispagestyle{empty}
\tableofcontents
\thispagestyle{empty}

\input Introducao
\input Latex
\input Modelos
\input Resultados
\input Discussao
\input Conclusao

%Referências
\newpage
\addcontentsline{toc}{section}{REFERÊNCIAS} %adiciona ao sumário
\renewcommand{\refname}{REFERÊNCIAS} %altera o nome da seção

\bibliography{Monografia} %arquivo que contém as referências

\end{document}